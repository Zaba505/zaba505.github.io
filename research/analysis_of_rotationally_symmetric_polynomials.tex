\documentclass{article}
\usepackage{amsthm}
\usepackage{amsmath}
\usepackage{amsfonts}
\title{Analysis of Rotationally Symmetric Polynomials}
\author{Richard Carson Derr (cakub6@gmx.com)}
\date{2022-12-16}
\newtheorem{definition}{Definition}
\newtheorem{lemma}{Lemma}
\newtheorem{remark}{Remark}
\newtheorem{theorem}{Theorem}
\begin{document}
\maketitle

\begin{abstract}
  Others have already studied the rotational symmetry
  of the graphs of polynomial function. There approach
  relied heavily on differential calculus. In this paper,
  we focus on developing an algebraic based framework
  to categorize the rotational symmetries of certain
  functions. Then, conclude with an application of this
  framework on the set of polynomial functions.
\end{abstract}

\begin{lemma}\label{negrot}
  Multiplication by -1 is equivalent to a rotation about the origin of $\pi$ radians.
\end{lemma}

\begin{proof}
  Let $\mathbf{F}$ be a field and $\begin{bmatrix}
    x \\
    y
  \end{bmatrix}
  \in
  \mathbf{F}^2$.
  Note, $\mathbf{F}^2$ is naturally a vector space over $\mathbf{F}$.

  \begin{equation}
    R_\theta
    \begin{bmatrix}
      x \\
      y
    \end{bmatrix}
    =
    \begin{bmatrix}
      cos(\theta) & -sin(\theta) \\
      sin(\theta) & cos(\theta)
    \end{bmatrix}
    \begin{bmatrix}
      x \\
      y
    \end{bmatrix}
    =
    -1
    \begin{bmatrix}
      x \\
      y
    \end{bmatrix}
  \end{equation}

  \begin{equation}
    \begin{matrix}
      x cos(\theta) - y sin(\theta) = -x\\
      x sin(\theta) + y cos(\theta) = -y
    \end{matrix}
  \end{equation}
  \begin{equation}
    \begin{matrix}
      cos(\theta) = -1 \\
      sin(\theta) = 0 \\
    \end{matrix}
  \end{equation}
  \begin{equation}\label{rottheta}
    \theta = (2n-1)\pi, \forall n \in \mathbb{N}
  \end{equation}

  Now, restricting n to 1 in Eq. \ref{rottheta}, we arrive at $\theta = \pi$, which
  gives us multiplication by -1 is equivalent to a rotation of $\pi$ radians.
  This restriction can be thought of intuitively as us defining multiplication by
  -1 as the smallest rotational increment allowed. Thus, successive iterations of
  multiplication would simply equate to $n$ multiples of $\theta$.
\end{proof}

\begin{definition}\label{odddef}
  Let $f$ be a function over a field, $\mathbf{F}$. f is considered "odd", if

  \begin{center}
    $\forall x \in \mathbf{F}$, f(-x) = -f(x).
  \end{center}
\end{definition}

\begin{definition}\label{graphdef}
  The graph of a function, $f:\mathbf{F} \rightarrow \mathbf{F}$, can be expressed as,

  \begin{center}
    $graph(f) = \{\,(x, f(x)) \mid x \in \mathbf{F}\,\}$
  \end{center}
\end{definition}

\begin{lemma}\label{oddrotlemma}
  All odd functions are rotationally symmetric.
\end{lemma}

\begin{proof}
  Let $f$ be an odd function over some field, $\mathbf{F}$. Then, by Def. \ref{graphdef} and the definition of a field,
  $(-x, f(-x)) \in graph(f)$. Also, note that $graph(f) \subseteq \mathbf{F}^2$, where
  $\mathbf{F}^2$ is naturally a vector space over $\mathbf{F}$. Thus,

  \begin{equation}
    \begin{split}
      (-x, f(-x))
      &= \begin{bmatrix}
        -x \\
        f(-x)
      \end{bmatrix} (\in \mathbf{F}^2)\\
      &= \begin{bmatrix}
        -x \\
        -f(x) \\
      \end{bmatrix} (by\ Def.\ \ref{odddef}) \\
      &= -1 * \begin{bmatrix}
        x \\
        f(x)
      \end{bmatrix} \\
      &= R_\pi \begin{bmatrix}
        x \\
        f(x)
      \end{bmatrix} (by\ Lemma\ \ref{negrot})
    \end{split}
  \end{equation}

  which provides the conclusion,

  \begin{center}
    $\forall p \in graph(f),\ \exists q \in graph(f)\ s.t.\ q = R_\pi p$
  \end{center}
\end{proof}

\begin{definition}\label{shiftdef}
  Let $f$ be a function over some field, $\mathbf{F}$. We then define the shift or
  translation operator, $T^t$, with the following functional equation,
  $T^t f(x) = f(x+t)$, where $t \in \mathbf{F}$.
\end{definition}

\begin{remark}\label{shiftsymremark}
  Shift operators preserve all symmetries of the function they're applied to.
\end{remark}

\begin{theorem}
  Let $\mathbb{F}$ be a set of functions over a field, $\mathbf{F}$. If $\forall f\in \mathbb{F}$, $\exists t \in \mathbf{F}$
  s.t. $T^t(f) = g, g(-x) = -g(x)$, then all $f \in F$ are rotationally symmetric.
\end{theorem}

\begin{proof}
  Let $f \in \mathbb{F}$ and suppose $\exists t \in \mathbf{F}$ such that

  \begin{center}
    $T^t(f) = g, g(-x) = -g(x)$
  \end{center}

  Then, by Lemma \ref{oddrotlemma}, $g$ is rotationally symmetric. Now, note that since $\mathbf{F}$ is a field,
  we can define another shift operator, $T^{-t}$, such that

  \begin{center}
    \begin{equation}\label{inverseshifteq}
      \begin{split}
        T^{-t}(g)
        &= T^{-t} T^t f(x) \\
        &= T^{-t} f(x+t) \\
        &= f(x+t-t) \\
        &= f(x)
      \end{split}
    \end{equation}
  \end{center}

  Finally, by Eq. \ref{inverseshifteq} and Remark \ref{shiftsymremark}, $f$ is rotationally symmetric.
\end{proof}

\end{document}