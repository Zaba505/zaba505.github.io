\documentclass{article}
\usepackage{amsthm}
\usepackage{amsmath}
\usepackage{amsfonts}
\title{Analysis of Rotationally Symmetric Functions}
\author{Richard Carson Derr (cakub6@gmx.com)}
\date{2022-12-16}
\newtheorem{definition}{Definition}
\newtheorem{lemma}{Lemma}
\newtheorem{remark}{Remark}
\newtheorem{theorem}{Theorem}
\begin{document}
\maketitle

\begin{abstract}
  Lorem ipsum
\end{abstract}

\begin{definition}\label{rotationoperator}
  We define the rotation operator in $\mathbf{R}^2$ as the following,
  \begin{equation}
    R_\theta
    =
    \begin{bmatrix}
      cos(\theta) & -sin(\theta) \\
      sin(\theta) & cos(\theta)
    \end{bmatrix}      
  \end{equation}
\end{definition}

\begin{lemma}
  \begin{equation}
    R_theta * R_phi = R_{\theta + \phi}
  \end{equation}
\end{lemma}

\begin{proof}
  \begin{equation}
    \begin{split}
      R_theta * R_phi
      &=
      \begin{bmatrix}
        cos(\theta) & -sin(\theta) \\
        sin(\theta) & cos(\theta)
      \end{bmatrix}
      \begin{bmatrix}
        cos(\phi) & -sin(\phi) \\
        sin(\phi) & cos(\phi)
      \end{bmatrix} \\
      &=
      \begin{bmatrix}
        cos(\theta)cos(\phi) - sin(\theta)sin(\phi) & -(cos(\theta)sin(\phi) + sin(\theta)cos(\phi)) \\
        cos(\theta)sin(\phi) + sin(\theta)cos(\phi) & cos(\theta)cos(\phi) - sin(\theta)sin(\phi)
      \end{bmatrix} \\
      &=
      \begin{bmatrix}
        cos(\theta + \phi) & -sin(\theta + \phi) \\
        sin(\theta + \phi) & cos(\theta + \phi)
      \end{bmatrix} \\
      &=
      R_{\theta + \phi}
    \end{split}
  \end{equation}
\end{proof}

\begin{lemma}
  \begin{equation}
    [R_theta, R_phi] = 0
  \end{equation}
\end{lemma}

\begin{proof}
  \begin{equation}
    \begin{split}
      [R_theta, R_phi]
      &= R_theta R_phi - R_phi R_theta \\
      &= R_{\theta + \phi} - R_{\phi + \theta} \\
      &= 0
    \end{split}
  \end{equation}
\end{proof}

\begin{remark}
  By the above, we observe the following:
  \begin{equation}
    R_theta R_{- \theta} = R_{- \theta} R_theta = R_{\theta - \theta} = R_0 = I
  \end{equation}
\end{remark}

\end{document}