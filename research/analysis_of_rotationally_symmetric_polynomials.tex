\documentclass{article}
\usepackage{amsthm}
\usepackage{amsmath}
\usepackage{amsfonts}
\title{Analysis of Rotationally Symmetric Polynomials}
\author{Richard Carson Derr (cakub6@gmx.com)}
\date{2022-12-16}
\newtheorem{definition}{Definition}
\newtheorem{lemma}{Lemma}
\newtheorem{theorem}{Theorem}
\begin{document}
\maketitle

\begin{abstract}
  Others have already studied the rotational symmetry
  of the graphs of polynomial function. There approach
  relied heavily on differential calculus. In this paper,
  we focus on developing an algebraic based framework
  to categorize the rotational symmetries of certain
  functions. Then, conclude with an application of this
  framework on the set of polynomial functions.
\end{abstract}

\begin{lemma}
  Multiplication by -1 is equivalent to a rotation about the origin of $\pi$ radians.
\end{lemma}

\begin{proof}
  $R_\theta$
  $\begin{bmatrix}
    x \\
    y
  \end{bmatrix}$
  =
  $\begin{bmatrix}
    cos(\theta) & -sin(\theta) \\
    sin(\theta) & cos(\theta)
  \end{bmatrix}$
  $\begin{bmatrix}
    x \\
    y
  \end{bmatrix}$
  =
  -1
  $\begin{bmatrix}
    x \\
    y
  \end{bmatrix}$

  $\implies$
  $\begin{matrix}
    x cos(\theta) - y sin(\theta) = -x\\
    x sin(\theta) + y cos(\theta) = -y
  \end{matrix}$

  $\implies$
  $\begin{matrix}
    cos(\theta) = -1 \\
    sin(\theta) = 0 \\
  \end{matrix}$

  $\implies$
  $\theta = (2n-1)\pi, \forall n \in \mathbb{N}$

  Now, restricting n to 1 so that we arrive at $\theta = \pi$, which
  gives us multiplication by -1 is equivalent to a rotation of $\pi$ radians.
  This restriction can be thought of intuitively as us defining multiplication by
  -1 as the smallest rotational increment allowed. Thus, successive iterations of
  multiplication would simply equate to $n$ multiples of $\theta$.
\end{proof}

\begin{definition}
  A function is considered "odd", if $\forall x \in \mathbb{R}$, f(-x) = f(x).
\end{definition}

\begin{lemma}
  All odd functions are rotationally symmetric.
\end{lemma}

\begin{definition}
  Let f be a function over $\mathbb{R}$. We then define the shift or translation operator, $T^t$, with the following
  functional equation, $T^t f(x) = f(x+t)$.
\end{definition}

\begin{theorem}
  Let F be a set of functions over $\mathbb{R}$. If $\forall f\in F$, $\exists T^t$
  s.t. $T^t(f) = g, g(-x) = -g(x)$, then all $f \in F$ are rotationally symmetric.
\end{theorem}

\end{document}